\documentclass[a4paper,12pt]{article}
%\documentclass[a4paper,12pt]{scrartcl}

\usepackage[utf8]{inputenc}

\title{Lesson 19. A very dear cat}
\author{}
\date{2019-10-28}

\pdfinfo{%
  /Title    (Lesson 19. A very dear cat)
  /Author   ()
  /Creator  ()
  /Producer ()
  /Subject  ()
  /Keywords ()
}

\begin{document}
\maketitle
Kidnappers are rarely interested in animals, but they recently took considerable interest in Mrs. Eleanor Ramsay's cat. Mrs. Eleanor Ramsay, a very wealthy old lady, has shared a flat with her cat, Rastus, for a great many years. Rastus leads an orderly life. He usually takes a short walk in the evenings and is always home by seven o'clock. One evening, however, he failed to arrive. Mrs. Ramsay got very worried. She looked everywhere for him but could not find him.

Three days after Rastus' disappearance, Mrs. Ramsay received an anonymous letter. The writer stated that Rastus was in safe hands and would be returned immediately if Mrs. Ramsay paid a ransom of \pounds 1,000. Mrs. Ramsay was instructed to place the money in a cardboard box and leave it outside her door. At first, she decided to go to the police, but fearing that she would never see Rastus again -- the letter had made that quite clear -- she changed her mind. She withdrew \pounds 1,000 from her bank and followed the Kidnapper's instructions. The next morning, the box had disappeared but Mrs. Ramsay was sure that the Kidnapper would keep his word. Sure enough, Rastus arrived punctually at seven o'clock that evening. He looked very well, though he was rather thirsty, for he drank half a bottle of milk. The police were astouned when Mrs. Ramsay told them what she had done. She explained that Rastus was very deat to her. Considering the amount she paid, he was dear int more ways than one!

\end{document}
